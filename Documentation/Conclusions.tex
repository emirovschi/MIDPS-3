\section*{Concluzie}
\phantomsection

Pentru efectuarea acestei lucrări am utilizat un set de instrumente care m-au ajutat la implementarea unui site. Am folosit git ca sistem de control al versiunilor. Acesta oferă posibilitatea de a înregistra fiecare modificare ca o versiune separată precum și crearea mai multor branch-uri. Ca mediu de dezvoltare am folosit IntelliJ care permite redactarea optimă a codului Java și are integrarea cu git și Maven. Pentru a controla structura și modul de construire a proiectului, precum și includerea eficientă și rapidă a dependențelor externe am folosit managerul Maven. Acest site este compus din 2 părți principale: Backend și Frontend. Prima este scrisă în Java și conține toată logica care se executa pe partea de server side. Una din dependențele principale utilizate aici este Spring framework care ușureaza crearea componentelor injectând automat toate dependențele, permite crearea unui REST API folosind controllere si metode anotate, oferă posibilitatea de a integra o bază de date într-un mod generic care ulterior permite să aplici orice tip de bază de date relațională. Pentru partea de frontend am folosit 2 framework-uri: AngularJS si Angular Material care conlucrează între ele. Primul oferă posibilitatea de a crea un single page application folosind MVW pattern. Angular Material oferă un set de utilități care ușurează procesul de a crea toate elementele din pagină și permite reutilizarea acestora cu scopul de a defini strucutra paginii pentru mai multe dispozitive. Toate aceste instrumente mi-au permis sa dezvolt eficient un produs software calitativ.

\clearpage